\documentclass[11pt,a4paper]{article}
%\usepackage{helvetica} % uses helvetica postscript font (download helvetica.sty)
%\usepackage{newcent}   % uses new century schoolbook postscript font  
\usepackage[utf8]{inputenc}
\usepackage[swedish]{babel}
\usepackage{lmodern}
\usepackage{amsmath}
\usepackage{amssymb}
\usepackage{tikz}
\usepackage{fancyhdr}
\usepackage{hyperref}
\def\thesubsection{\arabic{section}.\alph{subsection}}
\hoffset = -0.5in
\setlength{\textwidth}{6in} % increase text width 

\pagestyle{fancy}
\fancyhead[L]{Jonas Hugo - Kristofer Hansson}
\fancyhead[R]{Master-Thesis Proposal}
\begin{document}

\begin{titlepage}
\begin{center}

% Upper part of the page. The '~' is needed because \\
% only works if a paragraph has started.

\textsc{\LARGE Master-Thesis Planning Report}\\[1.5cm]

% Title
\huge \bfseries  A Generator of an Incremental Divide-and-Conquer Lexers\\[0.4cm]

\textsc{\Large }\\[0.5cm]


% Author and supervisor
\begin{minipage}{0.4\textwidth}
\begin{flushleft} \large
\emph{Groupmember 1:}\\
Jonas \textsc{Hugo}\\
MPALG\\
861017-5534
\end{flushleft}
\end{minipage}
\begin{minipage}{0.4\textwidth}
\begin{flushright} \large
\emph{Groupmember 2:} \\
Kristofer Hansson\\
MPALG\\
861208-4817
\end{flushright}
\end{minipage}
\vfill
% Bottom of the page
{\large \today}
\end{center}
\end{titlepage}

\section{Background}
Editors normally have regular-expression based parsers, which are efficient and
robust, but lack in precision: they are unable to recognize complex structures.
Parsers used in compilers are precise, but typically not robust: they fail to
recover after an error. They are also not efficient for editing purposes,
because they have to parse files from the beginning, even if the user makes
incremental changes to the input. More modern IDEs use compiler-strength
parsers, but they give delayed feedback to the user. Building a parser with good
characteristics is challenging: no system offers such a combination of
properties.

\section{Aim for the work}
The goal of the project is to develop a generic tool that can translate
any lexical specification into an incremental lexer for use in a syntax-aware
editor. Most programming environments provide
interactive feedback to the programmer. This feedback is often largely based on
syntactic analysis of the source code being edited. Instead of using syntactical
analysis this project will base the analysis on a lexer that can be run in real
time while the user modifies the text.

The project will be divided into three main parts, a generic tool which generate
an incremental lexer, and if time premits
integrate the result with an incremental parser. The tool for generating the
lexer will take precedence in the project. All the parts should fill the
following requirements:
\begin{description}
    \item[Robust]
            Errors in the input should be handled as gracefully as 
            possible. Ideally, errors in the input should only have local 
            effects on the lexer. 
    \item[Efficient]
            Fast feedback is important. If feedback is slow, either the user 
            will have to make frequent pauses, disrupting their work, or they 
            risk reacting to outdated, incorrect information.
    \item[Precise]
            The analysis performed by the parser should be as close as possible 
            to that made by the compiler.
\end{description}

\section{Limitations}
The project will not focus on the building of underlying structres as DFA.
It will neither focus on layout specific structure in a language. 
And for simplicity the lexer will only be tested for correcness on lexing the
java language.

\section{Method}
The project will be carriedout in the follwing steps.
\begin{itemize}
    \item 
        We will analyse parser descriptions and produced lexing tables suitable 
        for divide-and-conquer lexing
    \item 
        We will write a lexer which takes an input (provided as a finger tree)
        and the above tables, and produces a sequences of tokens corresponding
        to the input. 
    \item
        Because the lexer will use a divide-and-conquer strategy, it will be
        usable incrementally. That is, if the input is updated, only the results
        corresponding to the nodes of the tree containing the update will have
        to be recomputed.
    \item
        The results will be integrated in an existing lexer-generation tool such
        as Alex or BNFC
\end{itemize}

\section{Time plan}

\begin{tabular}{l l}
2013-06-17 - 2013-10-01 & Write report as project goes along\\
2013-06-17 - 2013-07-17 & Research in lexing techniques\\
2013-07-17 - 2013-08-20 & Develop the incremental lexer\\
2013-08-29 & Attend first oral presentations\\
2013-08-20 - 2013-09-02 & Prove correctness of developed lexer\\
2013-08-20 - 2013-09-09 & Preformence testing the lexer\\
2013-08-20 - 2013-10-01 & Focus on writing report\\
2013-10-01 $<$ & attend secound oral presentation\\
2013-10-01 $<$ & Opponent at another project\\
2013-10-01 $<$ & Oral presentation

\end{tabular}


\end{document}

