\chapter{Conclusion and Future Work}
As mentioned in the result chapter the incremental lexer was both robust and
precise. This means that without considering the time and space effiency an
incremental lexer will produce the same result as a sequential lexer with the
difference being how lexical errors are handled. The incremental lexer is
efficient in the sense that updates are done in $\Theta \log(n)$ time. However
when a tree is built up from scratch the incremental lexer takes
$\Theta n\log(n)$ time compared to the sequential lexer that takes $\Theta n$
time.

\section{Conclusions}
Incremental lexers is not suited to be used in a stand alone lexer since a
sequential lexer is more effiecent then an incremental lexer when an entire text
is being lexed. If a development environment that uses an incremental lexer was
used, the stand alone lexer can be omitted since the tokens are already
generated, saving one step in the compilation process.

It is however suited in an environment where updates are likely to happen, for
example to give lexical feedback in a text editor where each key stroke would be
an update. Insertion of a character in the text will be faster with an
incremental lexer compared to a sequential lexer since the lexical analysis does
not need to be done on the entire text. This means that the lexer could be run
in real time without a user noticing it. The result from an incremental lexer
can be passed to an incremental parser, giving parsing feedback to the user
instead. However, loading times when opening files will be longer if the tree
containing the tokens are not stored.

The space requirements for the incremental lexer grows with the tree. There is
information for the entire text in all levels of the tree and each level has
information for all possible in states. The space of a tree grows with
$\Theta mn\log(n)$, where $m$ is the number of states in the DFA. This means
that the memory usage will be big for large files and complex languages.

\section{Future Work}
To solve the problem with the space requirements for this implementation an
implementation using sequence of characters could be used instead. That is,
instead of using a character as the base case, a sequence of characters is used
which is sequentially lexed, an example of a sequence is one line of code. This
would shrink the tree from $\log(n)$ to $\log(n/x)$ where x is the mean length
of a line. Since lines in the code is roughly of the same length there will be
no impact on the worst case scenario time, for instance lexing 10 characters
always takes the same amount of time. Since the lines in general are short
updating a line will not take long time.

Another solution which could be used is to limit how big a tree can be. When
that text is bigger then what fits in a tree, the tree is split into two trees.
This will result in smaller trees at the expense of run time since the
combination of the trees needs to be calculated on the fly.

In general a lot of in states will have the same sequence of tokens. The
implementation suggested by this report will store all such sequences
separately. An improvement would be if somehow the sequences of tokens that are
identical could be stored in a separate table and the in state in the transition
map points to the corresponding sequence for that in state. This would not
improve the space complexity, but he practical space needed would shrink.
