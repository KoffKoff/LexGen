%\begin{filecontents*}{time.csv}
%characters*1000, time in microsecs
%n      alex   upd
%1000   210    150
%10000  1901   251
%20000  3910   329
%30000  5303   361
%40000  7079   361
%50000  8933   367
%60000  12140  460
%70000  14860  469
%80000  16850  523
%90000  17170  424
%100000 20920  554
%110000 24960  486
%120000 26430  503
%130000 28420  649
%140000 37890  575
%150000 33790  570
%160000 36850  624
%170000 39860  468
%180000 42570  496
%190000 42410  513
%\end{filecontents*}

\begin{filecontents*}{time.csv}
%characters*1000, time in microsecs
n       alex   upd  inc
1000    137    101  8492
10000   1372   174  202100
20000   2794   227  434100
30000   4103   236  498900
40000   5511   243  780100
50000   6935   228  935500
60000   8368   286  1000000
70000   9870   320  1452000
80000   11560  297  1479000
90000   13120  255  2105000
100000  14910  345  2086000
110000  16600  332  2037000
120000  18510  344  2478000
130000  20410  369  2578000
140000  22230  350  2617000
150000  24210  366  2668000
\end{filecontents*}

\begin{filecontents*}{space.csv}
n       size
100     11
200     21
400     40
800     77
1600    259
3200    387
6400    643
12800   1283
25600   2436
51200   4870
102400  9737
204800  19346
409600  38688
819200  77244
1638400 154486
\end{filecontents*}

\chapter{Result}
The incremental lexer has three requirements, it should be
Robust, Efficient and Precise. Robustness means that the lexer does not crash
when it encounter an error in the syntax. That is, if a string would yield an
error when lexed from the starting state the lexer does not return that error but
instead stores the error and lexes the rest of the possible input states since
the current string might not be at the start of the code. The implementation
this report purposes is robust since it stores errors in the data structure
rather then returning an error.

For it to be efficient the feedback to the user must be instant, or more
formally the combination of two strings should be handled in $\Theta(log(n))$ time.

Finally to be precise the lexer must give a correct result. This chapter is
describing how these requirements are tested and what the results are.

In the sections Below, any mention of a sequential lexer refers to a lexer
generated by Alex using the same alex file as is used when creating the
incremental lexer \cite{alex}. The reason why Alex is used is because the DFA
generated by Alex is used in the incremental lexer, thus ensuring that only the
lexical routines differs.
%\section{Robustness}

\section{Preciseness}
For an incremental lexer to work, the lexer must be able to do lexical analysis
of any sub text of a text and be able to combine two sub texts. If the lexical
analysis of one sub text does not result in any legal tokens it must be able to be
combined with other sub texts that makes it legal tokens. The lexical analysis of
a sub text might not always result in the same tokens that the combination of the
sub text with another text would give.

To test these cases a test is constructed which does a lexical analysis on two
sub texts using the incremental lexer and then combining the results into one
text. The result of the combination should be the same as the lexical analysis
of the text using the incremental lexer and the result using a sequential lexer.

It is not enough to test if the combination of two sub texts yields the
same sequence of tokens as the text. To test that the result of the
incremental lexer is the correct sequence of tokens generated, it is compared to
what a sequential lexer generates. This comparison is an equality test of
the text, it checks token for token that they are the same kind of token and
have the same lexeme.

\cref{fig:CheckEquility} shows the test for equality:
\begin{figure}[h!]
  \centering
  \lstinputlisting[language=c]{examples/CorrectEquility.hs}
  notEqual function is a function which pattern-match on the two different
  tokens and returns true if they are not of the same type.
  \caption{Code for testing tokens from IncLex is equal to tokens from Alex. 
  \label{fig:CheckEquility}}
\end{figure}

We performed tests on different files and cut the file in different places when
the update was tested. In all the tests the incremental lexer produced the same
token as the sequential lexer.

%\section{Efficiency}
\section{Performance}
To measure the performance of the incremental step two fingertrees are created,
each representing one half of a code. By creating the two fingertrees the
transition map for the code in those trees are created as well. The benchmarking
is then done on the combination of the two trees. The results of the incremental
lexer benchmarking suggests a running time of $\Theta(log(n))$. To get a reference
point the same text was lexed using a sequential lexer. The benchmarks can be
found in \cref{fig:IncSeqTime} and \cref{fig:IncTime}. More details on the
performance benchmarks can be found in \cref{preformanceAppendix}.

%Tester för nybyggning av träd?
\begin{figure}[!h]
%\begin{center}
\begin{tikzpicture}
    \begin{axis}[
    width=0.7\textwidth,
    height=0.4\textwidth,
    ymajorgrids,
    xmajorgrids,
    title = {Incremental lexer},
    scaled y ticks=real:1,
    scaled x ticks=real:1e3,
    ymin=0,
    xmin=0,
%    x tick label style={/pgf/number format/1000 sep=},
    ytick scale label code/.code={},
    xtick scale label code/.code={},
    ylabel={$\mu s$},
    xlabel={$*1000$ characters},
    legend pos=outer north east
    ]
        \addplot table[x=n,y=upd] {time.csv};\addlegendentry{Incremental update}
    \end{axis}
\end{tikzpicture}
%\end{center}
\caption{Benchmarking times of the incremental lexer\label{fig:IncTime}}
\end{figure}

\begin{figure}[!h]
%\begin{center}
\begin{tikzpicture}
    \begin{axis}[
    width=0.7\textwidth,
    height=0.4\textwidth,
    ymajorgrids,
    xmajorgrids,
    title = {Update compared to sequential},
    scaled y ticks=real:1e3,
    scaled x ticks=real:1e3,
    ymin=0,
    xmin=0,
%    x tick label style={/pgf/number format/1000 sep=},
    ytick scale label code/.code={},
    xtick scale label code/.code={},
    ylabel={$ms$},
    xlabel={$*1000$ characters},
    legend pos=outer north east
    ]
        \addplot table[x=n,y=upd] {time.csv};\addlegendentry{Incremental update}
        \addplot table[x=n,y=alex] {time.csv};\addlegendentry{Sequential lexing}
    \end{axis}
\end{tikzpicture}
%\end{center}
\caption{Comparison between an incremental update and sequential lexer\label{fig:IncSeqTime}}
\end{figure}

The running time when constructing the tree is not as fast as either the update
or the sequential lexer. The tests for the running time when constructing a new
tree suggest $\Theta n\log(n)$, this is expected since there are $\log(n)$
updates for every character. The result can be seen in \cref{fig:IncNewTime}.

\begin{figure}[!h]
%\begin{center}
\begin{tikzpicture}
    \begin{axis}[
    width=0.7\textwidth,
    height=0.4\textwidth,
    ymajorgrids,
    xmajorgrids,
    title = {Comparison},
    scaled y ticks=real:1e3,
    scaled x ticks=real:1e3,
    ymin=0,
    xmin=0,
%    x tick label style={/pgf/number format/1000 sep=},
    ytick scale label code/.code={},
    xtick scale label code/.code={},
    ylabel={$ms$},
    xlabel={$*1000$ characters},
    legend pos=outer north east
    ]
        \addplot table[x=n,y=inc] {time.csv};\addlegendentry{Incremental lexing}
        \addplot table[x=n,y=alex] {time.csv};\addlegendentry{Sequential lexing}
    \end{axis}
\end{tikzpicture}
%\end{center}
\caption{Comparison between the incremental and sequential lexer\label{fig:IncNewTime}}
\end{figure}

Since the incremental lexer uses a tree structure there will be an increase in
data-space used compared to a sequential lexer. Since the root of every subtree
stores the completed tokens for that tree, every level of the tree will store
approximately the size of the complete tokens for the entire text. A lexer that
uses a DFA with $m$ states will produce token data structure that grows with
$\Theta(mn)$ since each in state will have the tokens for that state. The depth
of the tree is $\log(n)$, this suggest a space complexity of
$\Theta(mn\log(n))$.

To test the space complexity a DFA for an early Java version that has 90 states
is used. To see how much the data structure grows the transition map was
serialized and stored to the disc using the Haskell library $Data.Binary$. The
test suggests that the size of the transition map grows linearly with the size
of text being lexed, the results can be found in \cref{fig:IncSpace}.

\begin{figure}[!h]
%\begin{center}
\begin{tikzpicture}
    \begin{axis}[
    width=0.7\textwidth,
    height=0.4\textwidth,
    ymajorgrids,
    xmajorgrids,
    title = {Space for transition map},
    ymode = log,
    xmode = log,
%    scaled y ticks=real:1,
%    scaled x ticks=real:1,
    x tick label style={/pgf/number format/1000 sep=},
%    ytick scale label code/.code={Bytes},
%    xtick scale label code/.code={Characters},
    ylabel={KB},
    xlabel={Characters},
    legend pos=outer north east
    ]
        \addplot table[x=n,y=size] {space.csv};\addlegendentry{Incremental lexer}
    \end{axis}
\end{tikzpicture}
%\end{center}
\caption{Space usage of the transition map using a DFA with 90 states\label{fig:IncSpace}}
\end{figure}
