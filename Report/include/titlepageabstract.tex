% Chalmers title page
\begin{titlepage}
\AddToShipoutPicture{\backgroundpic{-4}{56.7}{fig/auxiliary/frontpage}}
\mbox{\includegraphics[scale=0.8]{fig/auxiliary/siM.png}}
\vfill
\addtolength{\voffset}{2cm}
\begin{flushleft}
	{\noindent {\Huge A Generator of Incremental Divide-and-Conquer
Lexers} \\[0.5cm]
	\huge A Tool to Generate an Incremental Lexer from a Lexical Specification \\[0.5cm]
	\emph{\Large Master of Science Thesis [in the Program MPALG]} \\[.8cm]
	
	{\huge JONAS HUGO}\\[.8cm]
	{\huge KRISTOFER HANSSON}\\[.8cm]
	
    {\Large
	\textsc{Chalmers University of Technology} \\
	Department of Computer Science and Engineering \\
	Göteborg, Sweden,  \mydate\today\\
	} 
	}
\end{flushleft}

\newpage

\ClearShipoutPicture
\thispagestyle{empty}
\begin{flushleft}
    {\noindent{
The Author grants to Chalmers University of Technology and University of 
Gothenburg the non-exclusive right to publish the Work electronically and 
in a non-commercial purpose make it accessible on the Internet. 
The Author warrants that he/she is the author to the Work, and warrants 
that the Work does not contain text, pictures or other material that 
violates copyright law. \\[0.5cm]


The Author shall, when transferring the rights of the Work to a third party
(for example a publisher or a company), acknowledge the third party about
this agreement. If the Author has signed a copyright agreement with a third
party regarding the Work, the Author warrants hereby that he/she has obtained
any necessary permission from this third party to let Chalmers University of
Technology and University of Gothenburg store the Work electronically and
make it accessible on the Internet.\\[2cm]

A Generator of Incremental Divide-and-Conquer Lexers \\
A Tool to Generate an Incremental Lexer from a Lexical Specification\\

JONAS HUGO, \\
KRISTOFER HANSSON, \\[0.8cm]

© JONAS HUGO, \mydate\today.\\
© KRISTOFER HANSSON, \mydate\today.\\[0.5cm]

Examiner: BENGT NORDSTRÖM\\[0.5cm]

Chalmers University of Technology\\
University of Gothenburg\\
Department of Computer Science and Engineering\\
SE-412 96 Göteborg\\
Sweden\\
Telephone + 46 (0)31-772 1000\\[0.8cm]

Cover:\\
The image is a troll image, and will not be used in the final report.\\
%with page reference to detailed information in this essay.\\
[0.5cm]

Department of Computer Science and Engineering\\
Göteborg, Sweden \mydate\today

    }
    }

\end{flushleft}

\end{titlepage}
% End Chalmers title page

\thispagestyle{empty}
\newpage
\clearpage
\mbox{}
\newpage
\clearpage
\thispagestyle{empty}

\begin{abstract}
This report aims to present a way to do lexical analysis incrementally instead of
the present norm: sequential analysis. In a text editor, where updates are common, an
incremental lexer together with an incremental parser could be used to give
users real time parsing feedback. Previous work has proven that regular
expressions can be implemented incrementally \cite{blog}, we make use of these
findings in order to show that it can be expanded to a lexical analyzer. The
results in this report shows that an incremental lexer is efficient, it can do
an update in $\Theta \log n$ time which makes it suitable when updates are
common. In order for an incremental lexer to be applicable it has to be precise,
only correctly lexed tokens are relevant. It is required that an incremental
lexer is robust, a lexical error for a
partial result must be handled gracefully since it may not propagate to the
final result. To achieve incrementality a divide and conquer tree structure,
fingertrees, is used that stores the intermediate lexical results of all the
partial trees. When an update to the tree is made only the effected node and its
parents are updated. The state machine in the implementation is generated by
Alex since it is efficient and enables support for lexical analysis of different
languages. The report finishes with giving suggestions for improvements to the
drawbacks found during the work, The suggestions given are mainly for improving
space complexity. This report shows that an implementation of an incremental
lexer can be precise, efficient and robust.

\end{abstract}

\newpage
\thispagestyle{empty}
\clearpage
\mbox{}
\newpage
\clearpage
\thispagestyle{empty}
\section*{Acknowledgments}
We would like to take the chance of thanking our supervisor at department of
computer science, Jean-Philippe Bernardy, with which we have had a lot of
constructive discussions. We would also like to thank our examinator at the
department of computer science Bengt Nordström. Lastly we would like to thank
everyone that has helped proof read this report.\\[1cm]

\hfill Jonas Hugo \& Kristofer Hansson, Göteborg \mydate\today
%\newpage
%\clearpage
%\mbox{}
