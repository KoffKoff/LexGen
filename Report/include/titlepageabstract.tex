% Chalmers title page
\begin{titlepage}
\AddToShipoutPicture{\backgroundpic{-4}{56.7}{fig/auxiliary/frontpage}}
\mbox{\includegraphics[scale=0.8]{fig/auxiliary/siM.png}}
\vfill
\addtolength{\voffset}{2cm}
\begin{flushleft}
	{\noindent {\Huge A Generator of Incremental Divide-and-Conquer
Lexers} \\[0.5cm]
	\huge A Tool to Generate an Incremental Lexer from a Lexical Specification \\[0.5cm]
	\emph{\Large Master of Science Thesis [in the Program MPALG]} \\[.8cm]
	
	{\huge JONAS HUGO}\\[.8cm]
	{\huge KRISTOFER HANSSON}\\[.8cm]
	
    {\Large
	\textsc{Chalmers University of Technology} \\
	Department of Computer Science and Engineering \\
	Göteborg, Sweden,  \mydate\today\\
	} 
	}
\end{flushleft}

\newpage

\ClearShipoutPicture
\thispagestyle{empty}
\begin{flushleft}
    {\noindent{
The Author grants to Chalmers University of Technology and University of 
Gothenburg the non-exclusive right to publish the Work electronically and 
in a non-commercial purpose make it accessible on the Internet. 
The Author warrants that he/she is the author to the Work, and warrants 
that the Work does not contain text, pictures or other material that 
violates copyright law. \\[0.5cm]


The Author shall, when transferring the rights of the Work to a third party
(for example a publisher or a company), acknowledge the third party about
this agreement. If the Author has signed a copyright agreement with a third
party regarding the Work, the Author warrants hereby that he/she has obtained
any necessary permission from this third party to let Chalmers University of
Technology and University of Gothenburg store the Work electronically and
make it accessible on the Internet.\\[2cm]

A Generator of Incremental Divide-and-Conquer Lexers \\
A Tool to Generate an Incremental Lexer from a Lexical Specification\\

JONAS HUGO, \\
KRISTOFER HANSSON, \\[0.8cm]

© JONAS HUGO, \mydate\today.\\
© KRISTOFER HANSSON, \mydate\today.\\[0.5cm]

Examiner: BENGT NORDSTRÖM\\[0.5cm]

Chalmers University of Technology\\
University of Gothenburg\\
Department of Computer Science and Engineering\\
SE-412 96 Göteborg\\
Sweden\\
Telephone + 46 (0)31-772 1000\\[0.8cm]

Cover:\\
The image is a troll image, and will not be used in the final report.\\
%with page reference to detailed information in this essay.\\
[0.5cm]

Department of Computer Science and Engineering\\
Göteborg, Sweden \mydate\today

    }
    }

\end{flushleft}

\end{titlepage}
% End Chalmers title page

\thispagestyle{empty}
\newpage
\clearpage
\mbox{}
\newpage
\clearpage
\thispagestyle{empty}

\begin{abstract}
This report aims to provide a lexer that is incremental as an alternative to
todays sequential lexers. The results of the implementation described shows that
an incremental lexer can do an update in $\theta \log n$ time which makes it
suitable when updates are common. Because of this an incremental lexer together
with an incremental parser could be used to give a user parsing feedback in a
text editor or IDE. The incremental lexer is divide and conquer and uses
fingertrees to store the intermediate results. \#NEEDS MORE!!

\end{abstract}

\newpage
\thispagestyle{empty}
\clearpage
\mbox{}
\newpage
\clearpage
\thispagestyle{empty}
\section*{Acknowledgments}
We would like to take the chance of thanking our supervisor at department of
computer science, Jean-Philippe Bernardy, with which we have had a lot of
constructive discussions. We would also like to thank our examinator at the
department of computer science Bengt Nordström. lastly we would like to thank
everyone that has helped proof read this report.\\[1cm]

\hfill Jonas Hugo \& Kristofer Hansson, Göteborg \mydate\today
%\newpage
%\clearpage
%\mbox{}
