\usepackage[utf8]{inputenc}
\usepackage[T1]{fontenc}
\usepackage[english]{babel}
\usepackage[parfill]{parskip}
\usepackage{savesym}
\usepackage{amsmath}
\usepackage{amsthm}
\usepackage{amssymb}
\usepackage{ae}
\usepackage{icomma}
\usepackage{units}
\usepackage{color}
\usepackage{graphicx}
\usepackage{epstopdf}
\usepackage{subfigure}
\usepackage{bbm}
\usepackage{caption}
\usepackage[square, numbers, sort]{natbib}
\usepackage{multirow}
\usepackage{array}
\usepackage{geometry}
\usepackage{fancyhdr}
\usepackage{fncychap}
\usepackage[hyphens]{url}
\usepackage[breaklinks,pdfpagelabels=false]{hyperref}
\usepackage{lettrine}
\usepackage{eso-pic}
\usepackage{datetime}
\usepackage{natbib}
\usepackage{cleveref}
\usepackage{listings}
\usepackage{tikz}
\usepackage[linesnumbered,ruled,vlined]{algorithm2e}
\usepackage{filecontents}
\usepackage{pgfplots}
\usepackage{syntax}
\usepackage{pdfpages}
\usepackage{scrextend}
\savesymbol{AtBeginEnvironment}
\usepackage[nonumberlist,acronym]{glossaries}
\restoresymbol{Glossary}{AtBeginEnvironment}

\usetikzlibrary{arrows,automata,positioning,shadows}

% \newState{}{}{}{}
%
% #1: internal name,
% #2: (visible) label,
% #3: node properties (e.g. accepting, initial)
% #4: relative position (e.g. right of=/left of=/above of=/below of=)

\newcommand{\newState}[4]{\node[state,#3](#1)[#4]{#2};}

% \newTransition{}{}{}{}
% #1: source state (internal name),
% #2: target state (internal name)
% #3: guard/edge label
% #4: edge direction (loop below/above, bend right/left)
\newcommand{\newTransition}[4]{\path[->] (#1) edge [#4] node {#3} (#2); } 


\newcommand{\rd}{\ensuremath{\mathrm{d}}}
\newcommand{\id}{\ensuremath{\,\rd}}
\newcommand{\degC}{\ensuremath{\,\unit{^\circ C}}}

% Fancyheader shortcuts
\newcommand{\setdefaulthdr}{%
\fancyhead[L]{\slshape \rightmark}%
\fancyhead[R]{\slshape \leftmark}%
\fancyfoot[C]{\thepage}%
}
\newcommand{\setspecialhdr}{
\fancyhead{}
\fancyhead[LO]{\leftmark}%
\fancyhead[RE]{\rightmark}%
\fancyfoot[C]{\thepage}%
}

\newcommand{\mail}[1]{\href{mailto:#1}{\nolinkurl{#1}}}
\newcommand{\backgroundpic}[3]{%
	\put(#1,#2){
		\parbox[b][\paperheight]{\paperwidth}{%
			\centering
			\includegraphics[width=\paperwidth,height=\paperheight,keepaspectratio]{#3}
			\vfill
}}}

\newcommand{\qeda}{\hfill $\blacksquare$}
\newtheorem{theorem}{Theorem}[section]
\newtheorem{corollary}[theorem]{Corollary}
\newtheorem{lemma}[theorem]{Lemma}

\theoremstyle{remark}
\newtheorem{remark}[theorem]{Remark}

\theoremstyle{definition}
\newtheorem{definition}[theorem]{Definition}
\theoremstyle{definition}
\newtheorem{example}[theorem]{Example}


