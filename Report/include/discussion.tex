\chapter{Discussion}
\#Discuss discuss!!

One would think that haskells quickcheck would be a good way to generate in-data
for the lexer. Since quickcheck is built to generate good input for testing a
function for any arguments \cite{QuickCheck}. But the problem is not to test any
string representation, it is instead to test valid code segments and any
substring of this code segments. Also invalid pieces of text, to see that the
lexer informs the user of syntactical errors in these texts. To write a input
generator in quickcheck which would generate full code with all of its
components and all the different properties would have to high cost in
develop time for the outcome. It would be more time efficient to test the lexer
on several different code files. There for the testing of the incremental lexer
has not been done with the help of quickcheck.

\#possibility of using sequential lexing first time?

\#When is it advantagous with incremental lexing
